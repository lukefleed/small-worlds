\section{Detecting Small-Worldness}

As we have seen, many real technological, biological, social, and information networks fall into the broad class of \emph{small-world} networks, a middle ground between regular and random networks: they have high local clustering of elements, like regular networks, but also short path lengths between elements, like random networks. Membership of the small-world network class also implies that the corresponding systems have dynamic properties different from those of equivalent random or regular networks. \s

\nd However, the existing \emph{small-world} definition is a categorical one, and breaks the continuum of network topologies into the three classes of regular, random, and small-world networks, with the latter being the broadest. It is unclear to what extent the real-world systems in the small-world class have common network properties and to what specific point in the \emph{middle-ground} (between random and regular) a network generating model must be tuned to genuinely capture the topology of such systems. \s

\nd The current \emph{state of the art} algorithm in the field of small-world network analysis is based on the idea that small-world networks should have some topological structure, reflected by properties such as an high clustering coefficient. On the other hand, random networks (as the Erd\H{o}s-R\'enyi model) have no such structure and, usually, a low clustering coefficient. The current \emph{state of the art} algorithms can be empirically described in the following steps:

\begin{enumerate}
    \item Compute the average shortest path length $L$ and the average clustering coefficient $C$ of the target system.
    \item Create an ensemble of random networks with the same number of nodes and edges as the target system. Usually, the random networks are generated using the Erd\H{o}s-R\'enyi model.
    \item Compute the average shortest path length $L_r$ and the average clustering coefficient $C_r$ of each random network in the ensemble.
    \item Compute the normalized average shortest path length $\lambda := L/L_n$ and the normalized average clustering coefficient $\gamma := C/C_n$
    \item If $\lambda$ and $\gamma$ are close to 1, then the target system is a small-world network.
\end{enumerate}

\nd One of the problems with this interpretations is that we have no information on how the average shortest path scales with the network size. Specifically, a small-world network is defined to be a network where the typical distance $L$ between two randomly chosen nodes (the number of steps required) grows proportionally to the logarithm of the number of nodes $N$ in the network.
$$ L \propto N $$
But since we are working with a real-world network, there is no such thing as "same network with different number of nodes". So this definition, can't be applied in this case. \s

\nd Furthermore, let's try to take another approach. We can consider a definition of small-world network that it's not directly depend of $\gamma$ and $\lambda$, e.g:

\begin{center}
    \emph{A small-world network is a spatial network with added long-range connections}.
\end{center}

\nd Then we still cannot make robust implications as to whether such a definition is fulfilled just using $\gamma$ and $\lambda$ (or in fact other network measures). The interpretation of many studies assumes that all networks are a realization of the Watts-Strogatz model for some rewiring probability, which is not justified at all! We know many other network models, whose realizations are entirely different from the Watts-Strogatz model. \s

\nd The above method is not robust to measurement errors. Small errors when establishing a network from measurements suffice to make, e.g., a lattice look like a small-world network. See \cite{https://doi.org/10.48550/arxiv.1111.4570} and \cite{10.3389/fnhum.2016.00096}. \s
